\documentclass[12pt]{article}
\usepackage{amsmath}
\usepackage[margin = 1in]{geometry}
\usepackage{graphicx}
\usepackage{booktabs}
\usepackage{natbib}
% highlighting hyper links
\usepackage[colorlinks=true, citecolor=blue]{hyperref}



%% meta data

\title{Data analytics effects in Baseball}
\author{Chenyu Mu\\
  Department of Statistics\\
  University of Connecticut
}

\begin{document}

\maketitle

\begin{abstract}
  Over the past two decades, there has been a notable resurgence in the utilization of data analytics across professional sports, businesses, and governmental sectors.
  This article will delve into an in-depth analysis of the myriad factors that potentially impact baseball players while they are actively engaged in the game. 
  The study aims to explore and elucidate the diverse elements that might influence and shape the performance, strategies, and outcomes within the realm of baseball. 
  \end{abstract}


\section{Introduction}
\label{sec:intro}
Sport psychology, an interdisciplinary field integrating concepts from both psychology and sports science, has become a fundamental element in enhancing athletic performance across a spectrum of sports, baseball included.
Its primary focus is to comprehend and optimize the psychological components that affect an athlete's mindset, emotions, and conduct, ultimately aiming to elevate their performance.
Now, let's explore specific key domains within sport psychology that exert significant influence on baseball players.
Baseball was selected as the focal sport due to its comprehensive nature, allowing for an analysis of nearly every facet of the game, except for the mental readiness of the player upon entering the field. 
Previous studies in the field \citet*{Dalmass2018baseball} have provided valuable insights, and we seek to build upon their findings. 



\paragraph{Specific Aims}
We measure performance in these two dimensions: (1) the player's subjective assessment of their performance. 
(2) the statistical analysis of the player's actual performance. Consequently, the research inquiries encompass the following:
Does sports psychology have an effect on a player's physical performance?
Does sports psychology have an effect on the way a player feels about his performance?
Can sports psychology be used to ensure that a player is almost always giving his best performance?



\section{Data}
\label{sec:data}
The data is from UConn baseball 2023. Following are definitions we may use:
**AB** at bats 
**AVG**batting average 
**BB**bases on balls 
**ER**earned runs
**H**hits; holds
**K**killed 
**OBP**on base percentage



\section{Methods}
\label{sec:meth}
The study was designed as a two-sample experiment employing a differences-in-differences analysis approach.
Participants were categorized into three positional groups and further divided into either control or experimental groups. 
These results were evaluated as t-tests for differences in mean between groups.Method way\citep[e.g.,][]{Dalmass2018baseball}.



\section{Results}
\label{sec:resu}
These limitations encompassed various factors, including but not restricted to the sample size, the participants' in-season playing time,
and the number of at-bats and innings pitched. The relatively small sample size utilized in the study, which significantly impacted the practicality of the results.



\bibliography{../manuscript/refs}
\bibliographystyle{chicago}

\end{document}