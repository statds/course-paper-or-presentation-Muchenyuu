\documentclass[12pt]{article}
\usepackage{amsmath}
\usepackage[margin = 1in]{geometry}
\usepackage{graphicx}
\usepackage{booktabs}
\usepackage{natbib}
% highlighting hyper links
\usepackage[colorlinks=true, citecolor=blue]{hyperref}



%% meta data

\title{Proposal: Data analytics effects in major league baseball}
\author{Chenyu Mu\\
  Department of Statistics\\
  University of Connecticut
}

\begin{document}

\maketitle


\paragraph{Introduction}
Baseball, spanning from the Civil War era to the Civil Rights movement and continuing through various historical epochs,
serves as a mirror reflecting numerous facets of American life. This beloved sport is deeply intertwined with the nation's culture,
economics, and technological progress. It has the power to catalyze movements, nurture a sense of pride, and even contribute to the healing of cities.
The objective of this study aimed to assess the effectiveness of sports psychology. Baseball was selected as the focal sport due to its comprehensive nature,
allowing for an analysis of nearly every facet of the game, except for the mental readiness of the player upon entering the field. Previous studies 
in the field \citep[e.g.,][]{Dalmass2018baseball} have provided valuable insights, and we seek to build upon their findings. 



\paragraph{Specific Aims}
We measure performance in these two dimensions: (1) the player's subjective assessment of their performance. 
(2) the statistical analysis of the player's actual performance. Consequently, the research inquiries encompass the following:
Does sports psychology have an effect on a player's physical performance?
Does sports psychology have an effect on the way a player feels about his performance?
Can sports psychology be used to ensure that a player is almost always giving his best performance?



\paragraph{Data}
The data is from UConn baseball 2023. Following are definitions we may use:
**AB** at bats 
**AVG**batting average 
**BB**bases on balls 
**ER**earned runs
**H**hits; holds
**K**killed 
**OBP**on base percentage



\paragraph{Research Design and Methods}
The study was designed as a two-sample experiment employing a differences-in-differences analysis approach.
Participants were categorized into three positional groups and further divided into either control or experimental groups. 
These results were evaluated as t-tests for differences in mean between groups.Method way\citep[e.g.,][]{Dalmass2018baseball}.



\paragraph{Discussion}
These limitations encompassed various factors, including but not restricted to the sample size, the participants' in-season playing time,
and the number of at-bats and innings pitched. The relatively small sample size utilized in the study, which significantly impacted the practicality of the results.



\bibliography{../manuscript/refs}
\bibliographystyle{chicago}

\end{document}